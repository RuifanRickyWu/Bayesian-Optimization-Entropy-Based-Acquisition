\documentclass{article}

\usepackage[final]{neurips_2023}


\usepackage[utf8]{inputenc} % allow utf-8 input
\usepackage[T1]{fontenc}    % use 8-bit T1 fonts
\usepackage{hyperref}       % hyperlinks
\usepackage{url}            % simple URL typesetting
\usepackage{booktabs}       % professional-quality tables
\usepackage{amsfonts}       % blackboard math symbols
\usepackage{nicefrac}       % compact symbols for 1/2, etc.
\usepackage{microtype}      % microtypography
\usepackage{xcolor}         % colors


\title{
MIE498 -- Undergraduate Research Thesis Proposal \\
Bayesian Optimization using Information Gain Acquisition Functions
}

\author{
  Ruifan Wu \\
  Department of Mechanical \& Industrial Engineering\\
  University of Toronto\\
  \texttt{ruifan.wu@mail.utoronto.ca} \\
  \texttt{Student ID: 1006702926}
}


\begin{document}

\maketitle

\section{Introduction}
Bayesian optimization (BO) is a sequential optimization methodology for locating the global minima of a general nonlinear function that may be corrupted by noise.
The standard BO formulation incorporates the use of Gaussian Process (GP) models that are used to model the objective function.
GP models offer the benefit of including error statistics which can be interpreted to represent how certain the model is about it's prediction.
Additionally, BO incorporates the use of acquisition functions (search functions, infill criteria or policies) to determine a promising candidate point which can be used to evaluate the true objective function.
In this thesis project, we aim to investigate the performance of information gain matrics -- such as entropy search, predictive entropy search and max value entropy search -- when used as the specified acquisition function for BO.


\section{Objectives}
The objectives of this thesis project are

\begin{itemize}
    \item Construct entropy-based acquisition functions (i.e., Entropy Search, Predictive Entropy Search and Max Value Entropy Search) that can be used in a Bayesian optimization setting where we use the statistics from a GP posterior distribution to compute the search policy for ES, PES and/or MES.
    \item Compare the performance of these acquisition functions in a Bayesian Optimization setting against other common acquisition functions such as the probability of improvement, expected improvement, lower confidence bound and/or Thompson sampling.
    \item Extend this work to the setting of optimization under uncertainty if time permits where we have Gaussian process models for robustness metrics (optional).
\end{itemize}

\section{Description of Activities}

To achieve these objectives, I will be focusing on:

\begin{itemize}
    \item Complete Literature Review on

    \begin{itemize}
        \item Gaussian Process Modelling
        \item Acquisition Functions in Bayesian Optimization Modeling
    \end{itemize}

    \item Curate multiple datasets in different dimensions in Bayesian Optimization settings

    \item Developing implementations for various acquisition functions including
    \begin{itemize}
        \item Entropy-based acquisition functions
        \begin{itemize}
            \item Entropy Search
            \item Max-value Entropy Search
            \item Predictive Entropy Search
        \end{itemize}

        \item Common acquisition functions
        \begin{itemize}
            \item Probability of Improvement
            \item Expected Improvement
            \item Lower Confidence Bound
            \item Thompson Sampling
        \end{itemize}
    \end{itemize}

    \item Testing and evaluating different proposed methodologies in the curated datasets and explore ideas on improving entropy-based search’s performance and efficiency in normal and under uncertainty scenarios.
\end{itemize}

\section{Evaluation Breakdown and Percentage}

\begin{itemize}
    \item Review of the literature: 10\%
    \item Development and description of theory: 20\%
    \item Implementation: 20\%
    \item Evaluation: 25\%
    \item Summary of Finding: 25\%
\end{itemize}

\section{Work Schedule Breakdown}
January:
\begin{itemize}
    \item Onboarding with GPytorch and Pytorch by finishing GPyTorch Regression Tutorial
    \item Implement the $2D$ and $nD$ Gaussian process models in GPytorch environment.
\end{itemize}

February:
\begin{itemize}
    \item Implement and test information gain metrics that can be used together with GP models
    \item Design of experiments to compute the performance of the information gain metrics
\end{itemize}

March:
\begin{itemize}
    \item Finish any implementation.
    \item Carry out experiments
\end{itemize}

April:
\begin{itemize}
    \item Finalize thesis write-up and submit for department review.
\end{itemize}

\begin{table}[h!]
  \caption{Work Schedule}
  \label{work-schedule}
  \centering
  \begin{tabular}{ll}
    \toprule

    Deliverable/ Task & Date/ Time Needed \\

    \midrule

    Meeting & Weekly / 1-2 hrs \\
    Reading related materials & Weekly / 6 hrs \\
    Coding & Weekly / 5 hrs  \\
    Testing and evaluations & Weekly / 2 hrs \\
    Final Thesis Report & A week before the last day of lectures / 40 hrs \\

    \bottomrule
  \end{tabular}
\end{table}

For the semester: around 230 hours.

\end{document}